%\title{LaTeX Example: How to double space a document}
% Based on http://tex.stackexchange.com/questions/50894/how-do-i-double-space-my-abstract-in-latex
\documentclass[oneside,12pt]{book}
\usepackage{setspace}   %Allows double spacing with the \doublespacing command
\usepackage{fancyhdr}
\begin{document}

\title{Writing Samples} 
\date{July 2018}
\author{Shiva Bhusal}
\maketitle

\doublespacing
\chapter*{Nonfiction}
\section*{Home and the away}
I was brought up in a rural part of Nepal, and I spent about five years in the city of Kathmandu--the only metropolis in the country. Considering the North American and European standard, Kathmandu is not a big city; it is far smaller than Cleveland, far smaller than even Toledo. There are few tall buildings but there is not a single skyscraper. There is not metro railway. Once upon a time, there was a trolley bus service which no longer operates. Water supply is still a problem. Sometimes, a fifteen minute of commute takes an hour due to a traffic jam.

Yet, it is home to about 2 million people and by far the biggest city in Nepal.

I came to the city of Kathmandu in the summer of 2011, and for the first few months, I felt more as an alien, and less as someone who is a part of the city. I was new to the city, new to the peculiar surrounding, and it was my fault to expect the same sort of friendliness I experienced in my village abode. I was naive about the city life and sometimes, I even felt, I was living in an entirely different country. I was not quite prepared for what the city was to offer. It took me three more years to get prepared for such offerings and to finally realize, I was now a part of the city, and it was the place I could rightly call my home. 

To me, home is a place where someone’s heart and soul lies, and I had thought, it is more about the people, and less about the surrounding. I had thought, home is a place wherein every ten minutes of walk, I find someone I know. My definition of the homeland was rooted in my rural upbringings which no longer implies to the big cities like New York and Chicago and even the smaller cities like Cleveland and Toledo. I now strongly believe, home is more about the surrounding and less about the people. I feel, even if I visit the city of Kathmandu after fifty years, and I don’t find a single person of my acquaintance, I will still feel myself at home.

There were days when it was very difficult for me to adjust to a new place. But, after living my life as a wanderer for about seven years, it is now easier for me to adjust myself to new surroundings, and these days, within a few months of stay, I feel myself at home. That place of stay --to me-- becomes a friend whom I have met after a very long time, so long that I can't remember when I met him for the last time, the only thing I can assert is that he is my friend and I have met him somewhere, sometime. 

That might be one of the reasons why I feel like at home in the city of Bowling Green after about a year of stay. That might be one of the reasons why during the last summer, I felt like at home in the city of Cleveland even within a stay of a couple of months, and even if I was not that much fascinated with the city. All these places and times and surrounding are perhaps written in one’s destiny, and it is difficult for a man to choose a single place in the world, and consider that place his only home.

There are travelers who travel across places for amusement. There are historians who meditate deeply on the history of the big cities and link their findings with the evolution of entire human civilization. There are architects and urban planners who analyze the landscapes of the cities through some nerdy measures. I belong to none, and  whenever I travel to a new city, my observations are the observations of a common man’s eyes perplexed and dazzled with the landscapes of the metropolis and their grandeur.

When you travel across several cities and spend some months, you always find something in the city close to what you are as a person. In some sense or other, it becomes your friend, a companion, and a soul mate. You fall in love with the city, and for those who are lucky enough, the city loves them too, but for those who are unlucky, the city becomes a tale to be forgotten.

The modern-day people travel across countries and even continents. They live in the urban areas and in the rural landscapes. For wanderers who belong to nowhere, home is everywhere they have lived in, and away is the place where they have never been to. \\

\textit{Dec 2017}

\section*{How fiction shapes our life.}
For someone like me who grew up in a community where Nepali was the only means of oral communication, writing in English has always been an art of enigma and experiment. Before coming to the United States in 2016, my writing process in English was a form of translation. Ideas first came to my mind in Nepali, and before I wrote them down, my subconscious mind unknowingly translated those ideas into English. 
My fascination with “English” as a language was solely because of the books I read—especially the works of fiction. During my freshman and sophomore years at College, I was hugely fascinated with the works of Ernst Hemingway. One of the things I learned from the author of A Farewell to Arms was the ultimate simplicity —the art of conveying deepest realities of human existence using modest and unadorned prose.  

Books of non-fiction provide an eye to critically analyze and understand the world while fiction helps us fit ourselves into the shoes of the characters and experience their emotional and psychological tension which we would have otherwise never experienced in our life.

The exact story depicted in fiction may not be true, but it is not groundless and perfectly imaginary. Even the fantasy fictions have some indirect connections with the realities of human life. In other words, fiction is an “untrue” way of understanding the truth. 

I consider most of the religious texts depicting the lives of gods and goddesses to be fiction masterpieces beautifully woven with symbolism, faith, and mythology. It is debatable whether the stories are true, but it is certain that the same form of devilish, and goodish characters exist in human life. The sufferings of the characters in the religious texts are also inevitable in our own lives. The power of fiction in religious texts is such that the symbolism has now evolved into a strong form of belief, and the characters have now lived beyond the stories inside people’s consciousness. 

The oral tradition of storytelling has now been replaced by the digital form, but this art was very rich and viable in my village when I was a kid. Like other kids in my neighbors, I too grew up listening to folktales from my grandmother. In other parts of the world as well, the oral form of storytelling may have been a child’s first introduction to literature in the pre-digital era. I think, no any means in the world teaches a child the moral and ethical issues of life as good as the oral tradition of fiction. 

We all love stories and we all consume them in various forms including books, movies. television serials, and even in the form of daily narratives from our closest friends. The stories we consume have a big impact in our understanding of the world and the faith we believe on. 

Fiction opens a door of boundless perspectives on human life. Besides serving as a healthy form of entertainment, it also helps us understand another unfamiliar part of the world and find fascination in an entirely new language and culture. 

For instance, from V.S Naipaul’s A House for Mr. Viswas, one can understand the social consciousness of the people living a colonial life in an agricultural island. Fitzgerald and Hemingway’s works are enough to give us the glimpses of the jazz age and the lost generation respectively.  Khalid Husseini’s works take us to the life of people in the war-stricken Afghanistan. 

A good work of fiction never loses its significance and it is never confined within one geographical boundary, language, or chapter of history. The ancient tales of the Mahabharata, Iliad and Odyssey, Don Quixote and the works of Shakespeare are still relevant, and they never fail to justify their aesthetic superiority.

In a rather personal context of an Author’s writing process, Naipaul says, fiction never lies. I agree with his assertion because I too feel, a great work of fiction is the truest possible reflection of our own lives. \\

\textit{April 2018} \\\\\\\\\\\\\

\section*{Cricket in the time of crisis}
The field where we played cricket was so small we barely needed eleven men in a team. There were about five trees with thick trunks on the off side which did the job of two-three fieldsmen. There were no trees on the mid-wicket area. The players—who felt comfortable playing on the leg side—often hit the ball in the midwicket area. It was the longest boundary of the field, and equally dangerous. About 30 meters away from the midwicket area stood the house of the richest person in the village. If anyone went aerial, the ball traveled to the backyard of that house. There was a big probability of window panes being broken. There was also a decent probability that someone sitting on the balcony got hit on their head with a cricket ball. 

And one day, one of my friends hit an aerial shot on the midwicket area which had the sheer misfortune to crash one of the windows and break it. We didn’t know what to do for a moment, but then we realized running away without allowing anyone to notice our faces was the best idea. We ran away, we knew someone was following us, but we were sure nobody noticed our face. After that day, none of us showed up our face around that house for the next one-two weeks. The window pane got repaired in a couple of days, but I can’t remember if we ever played cricket in that field again. 

Our cricket field was a small compound of a primary school. It was as big as the 20-meter circle of the real cricket field. If there were more than five fieldsmen on the ground, as a batsman I felt everyone was around me, and it was almost impossible to score runs. Whoever went aerial over the fence or to the mustard field was declared “boundary out”.  We didn’t have money to buy new cricket balls, and every time the ball went to the mustard or cornfield, we searched for the ball until we found it. If we didn’t find the ball, we were done with the day. 

The situation was such that we had to leave the bouncers and even the balls that came to the leg stump. This made my batting skills overly defensive. I never learned how to play a hook shot, and I always struggled with the bouncers when we played cricket on the bigger grounds. I had a good defense but had no shots in the book except for a small jab behind the slip, cover drive and straight drive. 

The time when we played cricket the most was the time when the country was in crisis—during the time of people’s revolution in Nepal 2003/2004. Our schools were closed for about a month, and we had no other jobs during the day than playing cricket. Once we left the high school and started our university studies, cricket in the streets and small fields existed only in our memories. 

I am in my village after almost two years and I have been told, kids these days do not play in the streets and school compounds, they are busy with their cellphones and electronic devices. \\

\textit{Dec 2017}\\

\section*{Embracing cultural differences}
My first week in the United States was a cultural shock. Everything felt new to me including the style of conversations, the community I lived in, the type of food people consumed, dress sense and even the orientation of the bathrooms. Every time, I talked with a new person, I felt it difficult to decipher what they said. It was also difficult for me to convince them what I meant. Once I had a hard time convincing a gentleman at Walmart that I was looking for “Quarter” and not “Water”. I ended up taking a pronunciation class with the English department. 

I was raised in a traditional Hindu family, and there was always a religious and cultural purpose associated with what I did as a person. In our community back in Nepal, among the religious people, all the successes and failures in life are attributed to one’s fate.  Such attribution comes from the trust in God as the creator, moderator, and performer, and we, human beings serving as a medium to perform drama on God's will. People around the world follow different religions, but the similar belief and opinion about life appear in different forms and shapes everywhere.

I believe, the primary difference of culture, faith, and religion is depicted best in the food habits of people.  There is a certain restriction on the type of food people consume based on religion and faith. In our community back home, people consume mutton and chicken as the non-vegetarian foods. Pork and buff are also common among other communities, but, consumption of beef is taken as a legal offense. Unlike Nepal, beef consumption is common in the Middle East and in the West where most of the people follow religions other than Hinduism. 

Food is just an example. When people move from one part of the world to another, they come across the different social ecosystem, and they should adapt themselves to newer definitions of freedom, proper attire, and etiquette. For instance, going to the bars or getting drunk is not acceptable in the community I grew up, but it is common here in the United States after a certain age. Similarly, unlike the West, we have a conservative opinion towards sex in the East, and sex before marriage is considered a big taboo. Arranged marriages are common in which families of the girl and the boy decide whether the two should get married. Same-sex marriage is still considered a moral and legal taboo. 

To me, culture is a matter of habit. When a certain practice becomes prominent within a certain community for a long period of time, it becomes established as a norm. The attire, food consumption, and even myths and beliefs are based on that culture. The culture becomes so deep-rooted in people’s consciousness that most of the people are resistive to changes. It is always convenient for them to follow whatever they have been following throughout their life. 

However, with travel being an inevitable part of the modern people, one is always exposed to a different culture. In such a scenario, one should be respectful and adaptive to the culture other people follow. The world is always a better place to live in when we have reverential spaces for people of all cultures, languages, communities and sexual orientations. 

It may be difficult for anyone to adapt to a new culture, but as time passes by, everything starts looking familiar and known. The person becomes somewhat native to the new culture, but deep inside, they are always the same person they used to be in their homeland. If one spends their childhood and youth in one country, then throughout their life, they carry their habits and culture wherever they go.

That may be one of the reasons why even after about two years of stay in the United States, I can’t have a sound sleep until I have lentils, curry, and rice--the good old \textit{bhaat set}—as my dinner. That may be one of the reasons why I get excited whenever I find a Nepali restaurant in any cities I travel. \\

\textit{April 2018}

\chapter*{Fiction}
\section*{The man in the Picture}

When I was a kid, my grandfather told me stories before I went to sleep. His stories were the folklores borrowed from books like \textit{Panchatantra} and \textit{Nepali Dantyakatha Sangraha.} One day, he decided to share something new and unique, and he told me the story of King Rana Bahadur Shah. 

“King Rana Bahadur Shah was moody and sociopath,” My grandfather said, “He inherited only a few of the traits of his great father. Like most of the other kings, he was indulged in immoral affairs; unlike others, he was moody, lethargic and fragile.” Those days, I didn’t understand what immoral affairs exactly meant, but I knew they meant something “bad”. 

During the time of King Rana Bahadur Shah, decision making wasn’t based on the King’s sole judgment; it was rather the outcome of the couturiers’ persistent suggestions and persuasions. A king could decide of his own; but usually, he didn’t. My grandfather said, “A king may be bad, but he is a very powerful person, and we shouldn’t go against him.” 

I was in my tenth grade when I didn’t see my grandfather for weeks. When I asked my mother about him, she said, “He has been to the city for protest and won’t be back until everything is over.” In my village, we had people of all sorts — the revolutionists, the oppressors, supporters of monarchy and people of all other existing political beliefs. 

Once, there was a friendly football match between two neighboring villages. During the match, three people, covering their faces with black masks, came on to the ground and started firing blind gunshots. All the people—including the players and fans panicked and ran away to save their lives. The gunshots injured five players and killed one, and the shooters fled away before the police squad arrived. On the same night, we heard a bomb blast in the nearby police station. In the morning, we found that the police station was destroyed, with more than half of the police force dead and almost all of them severely injured. All the arms and weapons of the police station was also stolen. 

From that day, police and armies started coming to our houses and began to inquire. They took some of the youths to the stations and often tortured them. Some of them were killed, some got mysteriously disappeared, only a few of them came back, but with a gloomy face and weak body, which clearly indicated months of torture in hell. My father thus decided it was not safe to stay in the village and he went to India and didn’t returned for almost five years. 

After the bomb blast, my school was closed for almost a month, and the entire village was engulfed in a strange silence during that period. Nobody dared to talk about that incident with anyone. As police suspected groups, people avoided walking together during the day, and they never came out of their house during the night. Nobody knew, who killed the innocents during the football match, and who was behind the blasted bomb at the police station. 
One day, a group of ten-twelve police came to our house and as expected, said, “Hey! Where is your father? Tell him, we have some work.” Without waiting for my answer, they entered inside and investigated every item and found nothing susceptive.  After my grandfather told them that my father had been to India, they instead took my grandfather for the custody. 

There was no news of him for almost a week, and only two of us— me and my mother stayed at home. My mother often spent most of her time weeping
and moaning. She was completely devastated. I was a kid, but even I could understand what was going on. My grandfather returned home after a week, but he was not the same person again. He started feeling pain in the joints and he was now depressed and often feared of the simple items, and kept a knife below his pillow every time he went to sleep. 

His fear and gloom continued for almost five years until he got a chance to take part in the historic protest. Although he was almost seventy and had lost most of his youth and jubilance, the idea of protest rekindled the youth in him, and at the time he left home and decided to go to the city for some weeks, I had seen a big ambition in his face. He no longer felt pain in his joints, and he no longer had to take the diabetic pills. 

The wave of the historical revolution had shaken the whole nation and brought everyone together. For the first few days, the protest was peaceful and there was no any oppression from the police. After a couple of days, things turned sour and the authority imposed the curfew on certain areas of the city. The protesters protested in the places where curfew was not imposed, but they, including my grandfather, were tempted to go the restricted areas. It was the tenth day of the protest, and at around midday when the protesters had just stepped into the restricted zone, police started firing guns at them and they ran away for protection. During the chaos, my grandfather—who couldn’t run and protect himself like the other youths—was shot in the thigh and was immediately rushed to the hospital. Due to persistent bleeding, he was unconscious for few days, and when he could speak after a week, the authority had lost the battle against the protesters. The King had kneeled before the people. 

The success of the protest overshadowed the pain of the gunshot and the chronic diseases in my grandfather’s body. I had never seen his face that happy and content. After the success of the protest, my father returned from India and we were a happy family again. But, my grandfather’s health grew weaker and he had to take multiple medicines for his chronic diabetes, for the nerve problems he had after he came back from the police station years back, and for the problems caused due to the shot in the thigh during the protest. I too had just completed my high school, grown into a young man and could read stories of my own. My grandfather, who was now often confined to bed, still told me tales, and although most of them were the ones already told by him, I curiously listened to each one of them and enjoyed them as if I were listening to them for the first time. 

It was the month of January, and my grandfather began coughing and started vomiting blood, had multiple complications in his body and passed away at the hospital. I wasn’t with him during his last hours, but I met him the day before he died. That day, he had felt better and was joking with the doctors and sharing his experience of the protest. He was a happy man, full of enthusiasm. I believe, he didn’t have much compliments, and he was satisfied with how his life went about. 

It has been almost fifteen years since he died, and during this period, I along with my father and mother left the village and migrated to the city. I got married to the love of my life, and now I am blessed with a five-year-old son who just looked at the picture on the living room walls and asked “Daddy, who is this man?” 

It had been long since I was lost with my memories with grandpa. My son’s question suddenly reminded me of him. It reminded me of my childhood and the tales he used to tell me before I went to sleep, including the one of King Rana Bahadur Shah. I also remembered how he unknowingly discouraged me going against the authority, and how he eventually rejected his own belief. 

I told my son, “He was a great storyteller. He left us years back before you were born and these days, he lives among the clouds and is looking at us from the sky.” 

My answer was too philosophical for my innocent son who seemed puzzled at first but soon happened to ignore my answer, who then picked up the phone from the table, and started playing video games. \\

\textit{June 2018}


\linespread{0.6}
\chapter*{Poetry}

\section*{A promise broken}
Looking at the stars above our head and  \\
touching the sea with our hands, \\
we had promised to sail across together. \\
You took our promise as a child’s act and \\
left me alone on the seashore. \\
These days, I look at the white ships \\
bidding me goodbye, until they disappear \\ 
from my sight. I stare at the moon and \\
wait for the turn of the tides. \\\\\\\\

\textit{Nov 2016}

\section*{A father to his new-born daughter, on war}
In a couple of years, you shall ask me,\\
What is war? \\
How can I tell you?... In war, \\
there is blood, there are killings,\\
there is turmoil.\\\\
War is when your daddy and mummy fight\\
on who is going to change your diapers.\\
Winning a battle for us is the feeling\\
we had when you were just born, and\\
when you cried for the first time in your life.\\
It is a feeling we often share \\
when we are finally able to make you \\
asleep at midnight. Dear daughter,\\
You cry a lot, and most of the times,\\
we lose this battle against you.\\\\
A war has different meaning\\
to different people. It’s only the men\\
who are bigger and stronger than us \\
think of it as blood, turmoil and killings.\\
Your father is a small man and \\
he doesn’t understand \\
the bigger meanings of war.\\\\
Your eyes are flower-buds.\\
The world is bigger and \\
more confusing than \\
what you see through your eyes.\\
But it is only a matter of time.\\
You will soon grow old and wise and\\
understand yourself \\
what war means to them and \\
what it means to a small person like you.\\\\

\textit{Jan 2017}

\end{document}
